\documentclass[a4paper,14pt]{article}

\usepackage[utf8]{inputenc}
\usepackage[english]{babel}
\usepackage{graphicx, array, blindtext}
\usepackage[colorinlistoftodos]{todonotes}

\usepackage{enumitem}
\usepackage{amsmath}
\usepackage{amsthm}

\usepackage{nameref}
\usepackage{amssymb}
\usepackage{xcolor}
\usepackage{floatrow}
\usepackage{cancel}
\usepackage{fancyhdr}
\usepackage{graphicx}
\usepackage{verbatim}
\usepackage[document]{ragged2e}

\rhead{CS765 Assignment 1}
\usepackage{subcaption}
\usepackage{listings}


\usepackage{hyperref}

\begin{document}
\centering{

\title{\fontsize{150}{60}{CS765 Assignment 1}}

\author{
Prathamesh Pilkhane\\
Shashwat Garg \\
Vedang Asgaonkar}
}

\date{Spring 2023}
\maketitle

\justifying

% \tableofcontents
% \newpage

\justifying

\section*{Introduction}

DESCRIPTION ABOUT OUR WORK

\section{Questions}

\subsection{ What are the theoretical reasons of choosing the exponential distribution?}

The process of finding a block is a poisson process, due to the semantics of the Proof of Work and Hashing threshold requirements.

Because the exponential distribution represents the difference between the two events in the poisson process, it is appropriate to use the exponential distribution to simulate the inter-arrival time gap between the blocks

\subsection{ Why is the mean of $d_{ij}$ inversely related to $c_{ij}$ ? Give justification for this choice.}

The amount of time a packet has to spend at a node $i$ is proportional to the number of packets ahead of it, and inversely proportional to the rate at which node $i$ is dispatching them to node $j$.
Since the latter is inversely related to link speed $c_{ij}$, hence $d_{ij} \propto \frac{1}{c_{ij}}$.



\end{document}