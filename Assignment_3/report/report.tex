\documentclass[a4paper,14pt]{article}

\usepackage[utf8]{inputenc}
\usepackage[english]{babel}
\usepackage{graphicx, array, blindtext}
\usepackage[colorinlistoftodos]{todonotes}

\usepackage{enumitem}
\usepackage{amsmath}
\usepackage{amsthm}

\usepackage{nameref}
\usepackage{amssymb}
\usepackage{xcolor}
\usepackage{floatrow}
\usepackage{cancel}
\usepackage{fancyhdr}
\usepackage{graphicx}
\usepackage{verbatim}
\usepackage[document]{ragged2e}

\rhead{CS765 Assignment 3}
\usepackage{subcaption}
\usepackage{listings}


\usepackage{hyperref}

\begin{document}
\centering{

\title{\fontsize{150}{60}{CS765 Assignment 3}}

\author{
Prathamesh Pilkhane\\
Shashwat Garg \\
Vedang Asgaonkar}
}

\date{Spring 2023}
\maketitle

\justifying

% \tableofcontents
% \newpage

\justifying

\section*{Introduction}

Hello and Welcome to our CS765 Assignment 3. Here we implement a Dapp using Solidity, Ganache and Truffle.

\section{Code Flow}

\verb|client.py| uses python library \verb|web3| to connect to the ganache server. It calls the respective functions which have been implemented in solidity.\\\\
We use the python library \verb|networkx| to create a power-law degree distribution graph. It uses the Holme and Kim algorithm for growing graphs with powerlaw degree distribution.\\\\
We adjust the gas limit so as to remove the gas limit exceeded errors, both in \verb|client.py| as well as \verb|Payment.sol|\\\\
The remaining code in \verb|client.py| is simply just calling the required functions and storing the result values of the \verb|sendAmount| function.\\\\
In \verb|Payment.sol|, we implement the required 4 functions. Except \verb|sendAmount|, all others are pretty simple functions. In \verb|sendAmount|, we just need to implement a BFS in solidity, so it just costs a lot more in gas payments.



\section{Observation}

\subsection{Success Rate}




\end{document}